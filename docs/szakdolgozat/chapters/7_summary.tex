\Chapter{Összefoglalás}

A szakdolgozatom a kezdeti céljaimat meglátásom szerint kellően teljesítette. Ugyan nem merítettem ki maximálisan a felvetett témákat, de sok mindenre választ kaptam, amely egy bővebb látáskört biztosított a mesterséges intelligenciákról, valamint egy részletesebb képet adott a szoftverfejlesztés folyamatáról is.

A dolgozatom elején bemutattam a Splendor társasjátékot, valamint a különféle változatait és kiegészítőit is. Részleteztem az alapjáték elemeinek felépítését, azok játékmenet szempontjából való tulajdonságait és a szabályrendszert is. Ismertettem továbbá az általam meghatározott szabálymódosításokat összevetve az eredetivel.

A következő fejezetben a játék matematika modellje került bemutatásra. A folyamatábrával szemléltettem a társasjáték menetének felépítését, valamint részleteztem a logikai hátteret is.

Ezt követően részletesen bemutattam az általam szoftveresen megvalósított játékot. Sorra vettem a játék elemeinek elkészítését, a különböző játékszabályok megvalósítását, a grafikai felületek kivitelezését, valamint a különböző interakciók megalkotását is. Az osztályok felépítését és kapcsolatait osztálydiagramon prezentáltam, továbbá a refaktorálást követően alakuló osztályok felépítésének szemléltetésére is bemutattam a módosult diagramot.

Ezután az általam megvalósított mesterséges intelligenciákat részleteztem, amely során kifejtettem az eltérő működésüket. E mellet pedig demonstráltam a hozzájuk tartozó folyamatábrákat is, amelyek által remélhetőleg érthetőbbé vált a felépítésük.

Az utolsó fejezetben a mesterséges intelligenciákkal végzett szimulációkat és méréseket mutattam be. A méréseket annak érdekében végeztem el, hogy látható legyen, hogy miként változnak az adott játékok a különböző logikák egymás ellen való játéka során, valamint, hogy egyre "erősebb" algoritmusokat tudja létrehozni. A mesterséges intelligenciákkal kapcsolatos fejlesztéseim véleményem szerint eredményesnek tekinthetők. A szimulációk elvégzése után a módosított algoritmusaimat megvizsgálva tudtam informálódni az adott logikák hibáiból, így a folytatásra nézve volt lehetőségem az elkövetett hibákból tanulni. Ennek köszönhetően az ötödik intelligencia egy kellően erős ellenfélnek tekinthető, amelynek a gondolkozása nagy mértékben hasonlít az emberéhez, leszámítva néhány precizitást igénylő lépést.

Ahogy korábban is említettem ez a téma korántsem tekinthető kimerítettnek, hiszen az alapszabályok egésze nem lett megvalósítva, a mesterséges intelligencia is továbbfejleszthető, valamint a programkód refaktorálása sincs teljesen elvégezve. Ezek tekintetében elmondható, hogy a dolgozat témájában van lehetőség bőven a további fejlesztésekre, de az én dolgozatom ezen keretek között záródik.
