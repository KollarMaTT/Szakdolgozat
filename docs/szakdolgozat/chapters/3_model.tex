\Chapter{A játék matematikai modellje}

A játék véletlentől függő elemeket tartalmaz, viszont a játékosokra nézve teljes információs. A játék menete véges állapotgéppel modellezhető. A leszűkített szabályrendszerre vonatkozó játékmenetet \aref{fig:flowchart}. ábra mutatja be.

\begin{figure}[h]
\centering
\includegraphics[scale=0.42]{images/flowchart.png}
\caption{A játék folyamatábrája.}
\label{fig:flowchart}
\end{figure}

Ahogy a folyamatábrán is látható a játék köreinek felépítése viszonylag egyszerűen zajlik. Kezdetben kijelöljük az éppen soron következő játékost, majd megkezdődik a köre. Először eldönti, hogy kártyát vagy zsetont szeretne választani. Ha kártyát, akkor a kártyahúzás után vége a körének. Ha viszont zsetont, akkor két lehetőség áll a rendelkezésére: vagy hármat választ, egyet-egyet a különböző színekből, vagy pedig kettőt, de azonos színből. A két lehetőség valamelyikének végrehajtása után szintén véget ér a kör. Ezután megvizsgáljuk, hogy megvalósult-e a játék végét jelentő feltétel. Ha nem, akkor a folyamat elejére ugorva kiválasztjuk a másik játékost és megkezdődik a köre, amiben ugyanezekkel a lehetőségekkel rendelkezik. Ha pedig igen, akkor a játék befejeződik.

A játékos természetesen csak a számára elérhető kártyákból és zsetonokból választhat. Egy zseton akkor elérhető, ha abból legalább egy van a játéktéren, a kártya pedig akkor, ha a megvásárlásához szükséges zsetonok a birtokában vannak. Ezek a zsetonok lehetnek sima, korábban elvett zsetonok, vagy akár a kártyák korábbi megvásárlása nyomán megszerzett bónusz értékek.

A kártyalapok a játéktérre való elhelyezése véletlenszerűen történik az adott szintű kártyapakliknak megfelelően. A kártyák szimmetrikusan szerepelnek a paklikban, így ezek teljesen egyenértékűek a teljes kártyakészletre nézve. A játék előrehaladtával és a kártyalapok és zsetonok fogyásával értelemszerűen folyamatosan változik az adott játékos számára elérhető lehetőségeknek az értéke. Ezeket az értékeket különféle preferenciák alapján lehet meghatározni, mivel minden játékos az adott stílusában játszik. Lehet a lehető legmagasabb pontszámú kártyákra, lehet több kisebb értékű lapra gyűjteni. Lehet a játékos saját bónuszait figyelve megfelelően felépíteni egy jól működő mechanizmust, viszont lehet az ellenfél lépését megakadályozó taktikát is követni. Ez által nincsen rendszer (biztos nyerő stratégia), ami alapján pontosan meg tudjuk határozni, hogy egy adott játékállapotban az adott játékos számára mi a lehető legmegfelelőbb lépés.

