\Chapter{Szimulációk}

A megvalósított mesterséges intelligenciák erősségének összeméréséhez szimulációkra volt szükség. A fejezetben ezen vizsgálatok elvégzésének módját, a kapott eredményeket és értékelésüket láthatjuk.

\Section{Szimuláció program}

A szimulációk elvégzéséhez egy külön programot hoztam létre, amelyben a különböző AI-okat játszattam egymás ellen. Ezeknek a szimulációknak az eredményét kezdetben a konzolra írtam ki, hogy fel tudjam mérni a különböző AI-ok képességeit.

\SubSection{AI1 az AI2 ellen}

A szimulációk elvégeztével egyértelműen látszott, hogy a második sokkal eredményesebb mint az első, hiszen általánosan körülbelül a 85\%-át tudta megnyerni a lejátszott játékoknak.

Különböző vizsgálatokat futtattam a két AI játszmáinak kapcsán. Először azt vizsgáltam, hogy milyen a lejátszott meccsek végén a játékosok összpontszámának gyakoriság hisztogramja tízezer játék esetén (\ref{fig:scores1v2}. ábra). Az ábrán látható, hogy az eloszlása közel normálisnak tekinthető, a várható értéke 25.29, a minimuma 15, a maximuma pedig 33.

\begin{figure}[h]
\centering
\includegraphics[scale=0.7]{images/final_scores_AI1vsAI2.jpg}
\caption{Az első és második AI összpontszámainak eloszlása.}
\label{fig:scores1v2}
\end{figure}


Egy hasonló vizsgálat során, ugyanezen játékszám mellett a lejátszott meccsek köreinek számát tekintve vizsgáltam meg a gyakoriság hisztogramját (\ref{fig:rounds1v2}. ábra). Ezen ábrán is látható, hogy a körök száma is normálisnak tekinthető, várható értéke 63.86, minimuma 43 és a maximuma 89. Érdekes módon hatvanhét és hatvannyolc közé nem esett egy darab körszám sem.

\begin{figure}[h]
\centering
\includegraphics[scale=0.7]{images/round_number_hist_AI1vsAI2.jpg}
\caption{Az első és második AI köreinek eloszlása.}
\label{fig:rounds1v2}
\end{figure}

Végezetül pedig egy játékban a pontszámaiknak növekedését elemeztem. (\ref{fig:player_scores1v2}. ábra). Az ábrán jól látható, hogy mivel az első AI teljesen véletlenszerűen választ kártyákat, így a játék késői fázisában is csak alacsony szintű lapokat vásárol, pedig lehet, hogy elérhetőek lennének számára magasabb szintűek is. Ezen okból a második algoritmus főként a játék második felében kiemelkedően az első felé tud kerekedni.

\begin{figure}[h]
\centering
\includegraphics[scale=0.6]{images/player_points_AI1vsAI2.jpg}
\caption{Az első és második AI pontszámainak növekedése.}
\label{fig:player_scores1v2}
\end{figure}

\SubSection{AI2 az AI3 ellen}

A szimulációk elvégzése után láthatóvá vált, hogy a második algoritmus nagyobb arányban tudott nyerni, mint a harmadik. Körülbelül 74\%-ban nyert a második a harmadikkal szemben. Ebből az a következtetés vonható le, hogy a lapválasztást tekintve érdemesebb mindig a lehető legmagasabb szintű lapot választani a játék megnyerése érdekében. Az eredmények tekintetében a harmadik helyett továbbra is a második AI logikáját vettem alapul a következő algoritmus megvalósításakor a kártyaválasztásra nézve.

\SubSection{AI2 az AI4 ellen}

Ahogy elvégeztem a szimulációkat, kiderült, hogy a második logika ez esetben is nagyobb arányban tudott nyerni, mint a negyedik. Általánosan 57\%-kát tudta megnyerni a lejátszott meccseknek. Az, hogy ez az algoritmus nem hatékony abból fakad, hogy a játék előrehaladtával a késői fázisban is csak azt nézi, hogy melyik a számára legkönnyebben elérhető lap, így az alacsony szintűekre kezd el gyűjteni, nem pedig az értékesebb lapokra, így a pontszerzés szempontja a háttérbe szorul és a másik logika felé tud kerekedni.

A konklúzió, hogy a második algoritmus eredményesebb volt a harmadiktól és a negyediktől egyaránt. Ezáltal sem a kártyaválasztás, sem a zsetonválasztás terén nem sikerült előrelépni.

\SubSection{AI2 az AI5 ellen}

Az ötödik AI megvalósítása során figyelembe véve az előző próbálkozások sikertelenségét, igyekeztem azokból tanulva kiküszöbölni azok buktatóit, ezáltal a kártya-, és zsetonválasztás terén egyaránt előrelépést eredményezni a másodikhoz képest.

Az AI elkészítése után lefuttatva a szimulációkat kiderült, hogy a fejlesztéseim sikeresek voltak.
Az ötödik algoritmus körülbelül annyival erősebb, mint amennyivel a második volt az elsőtől. Általánosan 86\%-kát nyeri meg a lejátszott játékaiknak. Ez bizonyult tehát a legeredményesebb logikának a szimulációim során.

Az elsőhöz hasonlóan megnéztem, hogy miként alakul a játékok végén a játékosok összpontszámainak gyakoriság hisztogramja szintén tízezer játszma esetén (\ref{fig:scores2v5}. ábra). Az előzőhöz hasonlóan az eloszlás közel normálisnak tekinthető, a várható értéke 24.9, a minimuma 15, a maximuma pedig 33.

\begin{figure}[h]
\centering
\includegraphics[scale=0.7]{images/final_scores_AI2vsAI5.jpg}
\caption{A második és ötödik AI összpontszámainak eloszlása.}
\label{fig:scores2v5}
\end{figure}

A következő vizsgálat során tízezer játékszám mellett a lejátszott meccsek köreinek számát tekintve vizsgáltam meg a gyakoriság hisztogramját, ahogy korábban is tettem (\ref{fig:rounds2v5}. ábra). Az eloszlás lényegesen különbözik abból az okból, mivel a két mesterséges intelligencia erőssége jelentősen eltér.

\begin{figure}[h]
\centering
\includegraphics[scale=0.7]{images/round_number_hist_AI2vsAI5.jpg}
\caption{A második és ötödik AI köreinek eloszlása.}
\label{fig:rounds2v5}
\end{figure}

Ezután pedig az egy játékban való pontjaiknak a növekedését is szintén megvizsgáltam (\ref{fig:player_scores2v5}. ábra). Ezen az ábrán látható, hogy a második AI korábban tud elkezdeni pontot érő lapokat vásárolni, viszont észrevehető, hogy az ötödik egyből egy nagyobb értékűt vásárolt meg, amelyre a korábbi körökben gyűjtött zsetonokat és bónuszokat. Ezt követően pedig újra gyűjteni kezdett a következőre. Ezen logika mentén magasan felül tud kerekedni a második AI-on és az előző ilyen vizsgálathoz képest majdnem tíz körrel hamarabb be is tudta fejezni a játékot.

\begin{figure}[h]
\centering
\includegraphics[scale=0.6]{images/player_points_AI2vsAI5.jpg}
\caption{A második és ötödik AI pontszámainak növekedése.}
\label{fig:player_scores2v5}
\end{figure}

% TODO: Az ábrákból itt is le kellene vonni valamilyen következtetést.

% TODO: Az eloszlásokra vonatkozóan érdekes lehet valamilyen egyszerűbb vizsgálat.

\SubSection{Hatékonyság}

Végezetül összegyűjtöttem egy táblázatba a különböző AI-ok hatékonyságát az egymással futtatott szimulációk során. Az eredmények \aref{tab:ai_comparison}. táblázatban láthatók az eredmények, amely elkészítéséhez olyan módon módosítottam a kódomat, hogy mindig egy adott algoritmus kezdte a játékokat. A sorok mutatják, hogy melyik AI kezdett, valamint, hogy milyen százalékban tudta megnyerni a játszmákat a többivel összemérve.

\begin{table}[h]
\caption{A gépi intelligenciák eredményessége az egymás elleni szimulációkban.}
\label{tab:ai_comparison}
\medskip
\centering
\begin{tabular}{|c|c|c|c|c|c|} 
 \hline
  & AI1 & AI2 & AI3 & AI4 & AI5 \\ 
 \hline
 AI1 & 55\% & 18\% & 40\% & 22\% & 5.5\%\\ 
 \hline
 AI2 & 86\% & 56\% & 78\% & 62\% & 17\%\\ 
 \hline
 AI3 & 67\% & 30\% & 55\% & 35\% & 8.5\%\\ 
 \hline
 AI4 & 85\% & 50\% & 74\% & 56\% & 12\%\\ 
 \hline
 AI5 & 96\% & 88\% & 95\% & 92\% & 52\%\\
 \hline
\end{tabular}
\end{table}

A tesztek elvégzése után az a következtetés vonható le, hogy a körkezdésnek viszonylag nagy jelentősége van a nyerési arány tekintetében, hiszen ahogy az első algoritmusnál is láthatjuk körülbelül öt százalékkal eredményesebb volt önmagával szemben az, amelyik kezdett. Ez a randomizálás visszaszorulásával arányosan csökken, ha megnézzük az ötödik AI esetét, ahol nagyobb szerepet kap a következetesség. Ebben az esetben már csak két százalékkal volt jobb a körkezdő algoritmus.

A második és a negyedik logika erősségének összemérése esetén ez a jelenség meglepően befolyásolta az eredményt. A korábbi tesztek során kiderült, hogy a második erősebbnek bizonyult, mint a negyedik AI. Ezzel szemben azokban az esetekben, amikor a negyedik kezdett, körülbelül azonos volt a nyerési arányuk.

A táblázat alapján nyerési átlagukból kiszámítva felállítottam egy erősségi sorrendet a különböző AI-ok között. \Aref{tab:ai_ranking}. táblázat fentről lefelé csökkenő sorrendben mutatja a erősorrendet, a jobb oszlopban pedig az látható, hogy az aktuális logika átlagosan milyen arányban tudott nyerni a különböző algoritmusokkal szemben. A \textit{(K)} jelzés azt az esetet jelöli, amikor az adott AI kezdte a játszmákat.

\begin{table}[h]
\caption{A gépi intelligenciák erősorrendje. (A zárójelben a K jelzi, hogy az adott AI kezdi-e a játékot.)}
\label{tab:ai_ranking}
\medskip
\centering
\begin{tabular}{|l|c|} 
\hline
Név & Átlag \\
 \hline
 AI5 (K) & 93\%\\ 
 \hline
 AI5 & 90\%\\ 
 \hline
 AI2 (K) & 61\%\\ 
 \hline
 AI4 (K) & 55\%\\ 
 \hline
 AI2 & 53\%\\
 \hline
  AI4 & 47\%\\ 
 \hline
 AI3 (K) & 35\%\\ 
 \hline
 AI3 & 28\%\\ 
 \hline
 AI1 (K) & 22\%\\ 
 \hline
 AI1 & 16\%\\
 \hline
\end{tabular}
\end{table}