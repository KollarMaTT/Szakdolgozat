\Chapter{Az AI megvalósítása}

A számítógép által vezérelt játékosokat különböző döntési logikák segítségével valósítottam meg.

\Section{Első AI}
Az első AI egy nagyon alapvető logikának a megvalósítása. A köre elején megnézi, hogy van-e számára elérhető kártyalap. Amennyiben igen, azok közül random módon választ egyet. Ha viszont nem, akkor a zsetonok közül választ szintén véletlenszerűen a megfelelő szabályok betartásával.

\Section{Második AI}
A következő AI egy szinttel volt csak a korábbinál komplikáltabb, ami a kártyaválasztás terén lépett egy szintet. Ez esetben, amikor van számára elérhető lap, akkor azok közül megnézi, hogy melyik vagy melyikek a legmagasabb szintűek, majd azok közül választ véletlenszerűen egyet. Abban az esetben, amikor nem volt elérhető lap, random módon választ zsetont.

\Section{Harmadik AI}
A harmadik AI a másodiknak egy változata, amelyben azt próbáltam megvalósítani, hogy ne feltétlen a lehető legmagasabb szintű lapot vegye meg az elérhetőek közül, hanem adott esetben válasszon kisebb szintű, de kevesebb költségű lapokat is. Ezt úgy eszközöltem, hogy lapválasztás esetén először megnézi, hogy van-e olyan számára megvásárolható kártya, amelynek a pontszáma kettővel osztható. Ha van, akkor azok közül választ véletlenszerűen, ha pedig nincs, akkor a második AI logikáját követve a lehető legmagasabb szintű lapok halmazából választ random módon. A zsetonválasztást nem módosítottam.

\newpage

\Section{Negyedik AI}
A negyedik AI egy olyan logikát foglal magába, amely a kártyaválasztás esetén a második AI logikáját követi, viszont a zsetonválasztásnál már figyelembe veszi a játékállapotot mind a lent lévő lapok, mind a rendelkezésére álló zsetonok tekintetében. Amennyiben nincs elérhető lap, az algoritmus kiválasztja a lent lévő lapok közül azt, amelyikhez a lehető legkevesebb zsetonra van szüksége. Ezután megnézi, hogy melyek a szükséges zsetonok, és ha valamelyik elérhető, akkor megvásárolja közülük az egyiket és folytatja a folyamatot a lépések szabályainak betartásával. Ezzel szemben, ha nem elérhető számára egy szükséges zseton sem, akkor a korábbiakhoz hasonlóan véletlenszerűen választ az elérhetőek közül.

\Section{Ötödik AI}
Az utolsó AI logikája a legkomplikáltabb és egyben a legeredményesebb a korábbiakkal összevetve. A korábbiaktól eltérőenmind a kártya-, mind a zsetonválasztás esetében figyelembe veszi, hogy mire lenne szüksége a hármas szintű lapok megvásárlásához, ezáltal egyfajta előre gondolkodást valósít meg.\par
A kör kezdetén kiválasztja, hogy a harmadik szintű lapokból melyik az, amelyhez a lehető legkevesebb zsetonra van szüksége és amennyiben ez elérhető számára megvásárolja.\par
Ha viszont nem, akkor megvizsgálja, hogy melyek a megvásárolható lapok. Abban az esetben, ha a megvehető lapok között van olyan, aminek a bónusza szükséges a kinézett lap megvásárlásához hiányzó értékekből, véletlenszerűen megveszi valamelyiket. Ha nincs ilyen lap, akkor megnézi, hogy van-e elérhető zseton. Hogyha van, akkor kiválogatja azokat, amik még szükségesek a kívánt lap megvásárlásához és ezekből random módon választ. Ha nincs ilyen zseton, akkor véletlenszerűen vesz egyet az elérhetőek közül. Ha egyáltalán nem elérhető zseton csak a korábban kiszűrt és a kívánt lap elérését nem elősegítő kártyák, akkor ezek közül vásárol random módon.\par
Abban az esetben viszont, amikor a kör elején nincs elérhető lap, akkor az imént kifejtett zsetonválasztási lépéseket hatja végre az algoritmus.