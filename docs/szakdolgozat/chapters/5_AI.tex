\Chapter{Az AI megvalósítása}

A számítógép által vezérelt játékosokat különböző döntési logikák segítségével valósítottam meg. A fejezetben ezek működésének, sajátosságainak a bemutatására kerül majd sor.

A különböző AI-ok a \texttt{Board} osztály játékállapotát egy \texttt{gameState} változón keresztül kapják meg mint bemenetet. Ezen változó segítségével tud informálódni a mesterséges intelligencia a játék aktuális állásáról, és ez alapján hajtja végre a számára megfelelő lépést. A kártya és zseton kiválasztása szintén a \texttt{Board} osztálytól kapott \texttt{buyCard} és \texttt{buyToken} metódusok segítségével történik. Ezek a metódusok az AI-ok logikája által meghatározott kártya vagy zseton adatait kapják meg és elvégzik a megfelelő módosításokat a játéktéren. Ez a két metódus tekinthető a mesterséges intelligenciák által végrehajtható akcióknak.

Azért, hogy az emberi játékos lépésenként nyomon tudja követni, hogy mit lép a gépi ellenfél, annak léptetéséhez tetszőleges helyre kell kattintani a programban.

A mesterséges intelligenciák nevében sorszám szerepel. Konkrétan az AI1, AI2, AI3, AI4 és AI5 nevű intelligenciák részletezésére kerül sor a következő szakaszokban.

\Section{AI1}

Az első AI egy nagyon alapvető logikának a megvalósítása. A köre elején megnézi, hogy van-e számára elérhető kártyalap. Amennyiben igen, azok közül véletlenszerű módon választ egyet. Ha viszont nem, akkor a zsetonok közül választ szintén véletlenszerűen a játék szabályainak betartásával.

Az AI1 folyamatábrája a függelékben megtekinthető (\ref{fig:AI1_flowchart}. ábra).


\Section{AI2}

A következő AI csak egy fokkal komplikáltabb a korábbinál, amely a kártyaválasztás terén lép egy szintet. Ez esetben, amikor van számára elérhető lap, akkor azok közül megnézi, hogy melyik vagy melyek a legmagasabb szintűek, majd azok közül választ véletlenszerűen egyet. Abban az esetben, amikor nincs elérhető lap, véletlenszerű módon választ zsetont.

Az AI2 folyamatábrája a függelékben megtekinthető (\ref{fig:AI2_flowchart}. ábra).


\Section{AI3}

A harmadik AI a másodiknak egy változata, amelyben azt próbáltam megvalósítani, hogy ne feltétlen a lehető legmagasabb szintű lapot vegye meg az elérhetőek közül, hanem adott esetben válasszon kisebb szintű, de kevesebb költségű lapokat is. Ezt úgy eszközöltem, hogy lapválasztás esetén először megnézi, hogy van-e olyan számára megvásárolható kártya, amelynek a pontszáma kettővel osztható. Ha van, akkor azok közül választ véletlenszerűen, ha pedig nincs, akkor a második AI logikáját követve a lehető legmagasabb szintű lapok halmazából választ véletlenszerűen egyet. A zsetonválasztást nem módosítottam.

Az AI3 folyamatábrája a függelékben megtekinthető (\ref{fig:AI3_flowchart}. ábra).


\Section{AI4}

A negyedik AI egy olyan logikát foglal magába, amely a kártyaválasztás esetén a második AI logikáját követi, viszont a zsetonválasztásnál már figyelembe veszi a játékállapotot mind a lent lévő lapok, mind a rendelkezésére álló zsetonok tekintetében. Amennyiben nincs elérhető lap, az algoritmus kiválasztja a lent lévő lapok közül azt, amelyikhez a lehető legkevesebb zsetonra van szüksége. Ezután megnézi, hogy melyek a szükséges zsetonok, és ha valamelyik elérhető, akkor megvásárolja közülük az egyiket és folytatja a folyamatot a lépések szabályainak betartásával. Ezzel szemben, ha nem elérhető számára egy szükséges zseton sem, akkor a korábbiakhoz hasonlóan véletlenszerűen választ az elérhetőek közül.

Az AI4 folyamatábrája a függelékben megtekinthető (\ref{fig:AI4_flowchart}. ábra).

\Section{AI5}

Az utolsó AI logikája a legkomplikáltabb és egyben a legeredményesebb a korábbiakkal összevetve. Az előző négy verziótól eltérően mind a kártya-, mind a zsetonválasztás esetében figyelembe veszi, hogy mire lenne szüksége a hármas szintű lapok megvásárlásához, ezáltal egyfajta előre gondolkodást valósít meg.

A kör kezdetén kiválasztja, hogy a harmadik szintű lapokból melyik az, amelyhez a lehető legkevesebb zsetonra van szüksége és amennyiben ez elérhető számára megvásárolja.

Ha viszont nem szerezhető meg, akkor megvizsgálja, hogy melyek a megvásárolható lapok. Abban az esetben, ha ezen lapok között van olyan, aminek a bónusza szükséges a kinézett lap megvásárlásához, véletlenszerűen megveszi valamelyiket. Ha nincs ilyen lap, akkor megnézi, hogy van-e elérhető zseton. Hogyha van, akkor kiválogatja azokat, amik még szükségesek a kívánt lap megvásárlásához és ezekből véletlenszerűen választ. Ha nincs ilyen zseton, akkor véletlenszerűen vesz egyet az elérhetőek közül. Ha egyáltalán nem elérhető zseton csak a korábban kiszűrt és a kívánt lap elérését nem elősegítő kártyák, akkor ezek közül vásárol szintén véletlenszerű módon.


Abban az esetben viszont, amikor a kör elején nincs elérhető lap, akkor az imént kifejtett zsetonválasztási lépéseket hatja végre az algoritmus.

Az AI5 folyamatábrája a függelékben megtekinthető (\ref{fig:AI5_flowchart}. ábra).

\Section{Mesterséges intelligenciák áttekintése}

\Aref{tab:ai_features}. táblázat összegzi az elkészített mesterséges intelligenciák főbb tulajdonságait.

\begin{table}[h]
\caption{A gépi intelligenciák működése az egymáshoz való viszonyítás során.}
\label{tab:ai_features}
\medskip
\centering
%\begin{tabular}{|m{2em}|m{7em}|m{7em}|m{7em}|m{7em}|}
\begin{tabular}{|c|c|c|c|c|} 
 \hline& Randomizált & Randomizált & Összetett  & Összetett  \\
&  kártyaválasztás &  zsetonválasztás &  kártyaválasztás  &  zsetonválasztás \\
 \hline
 AI1 & \checkmark & \checkmark & $\times$ & $\times$ \\ 
 \hline
 AI2 & $\times$ & \checkmark & \checkmark & $\times$ \\ 
 \hline
 AI3 & $\times$ & \checkmark & \checkmark & $\times$ \\ 
 \hline
 AI4 & $\times$ & $\times$ & \checkmark & \checkmark \\ 
 \hline
 AI5 & $\times$ & $\times$ & \checkmark & \checkmark \\
 \hline
\end{tabular}
\end{table}