\Chapter{Bevezetés}

A szakdolgozatom a Splendor elnevezésű társasjáték általam módosított verziójának szoftveres megvalósításáról, a játékon belül található mesterséges intelligenciáról, valamint az ehhez kapcsolódó vizsgálatokról szól. Azért választottam ezt a témát, mert a társasjátékok világa mindig is közel állt hozzám, ezenfelül pedig a mesterséges intelligenciák köre is hasonló módon foglalkoztatott. Ennek a két területnek az ötvözésével egyértelművé vált számomra, hogy egy társasjátékkal foglalkozó dolgozatot szeretnék elkészíteni, valamint kutatásokat elvégezni azzal kapcsolatosan, hogy hogyan érdemes megvalósítani egy intelligenciát, hogy az a lehető leghatékonyabb legyen.

A második fejezetben bemutatásra kerül maga a társasjáték és a különböző fizikális és szoftveres változatai. A Splendor szabályrendszerét ismertetem a különböző játékelemek és az őket behatároló szabályok kifejtésével. Ezek mellett szerepel az általam elkészített változat szabályrendszere, a játékelemekben való eltérések.

A harmadik fejezetben a játék matematikai modelljét mutatom be, amely során kitérek a társas logikai hátterére és egy folyamatábra segítségével szemléltetem a játék folyamatának felépítését.

A negyedik fejezet az általam szoftveresen megvalósított játékról szól. Prezentálom, hogy a kezdeti terveimhez képest a játék mely részei módosultak a megvalósítás során, valamint a különböző játékelemek, szabályok és grafikai megvalósítások kivitelezését is részletezem. Ezenfelül a játék osztálydiagramját is bemutatom.

Az ötödik fejezetben az általam megvalósított öt mesterséges intelligencia kerül bemutatásra. Ismertetem a működésüket, felépítésüket, valamint mindegyik algoritmushoz bemutatom a folyamatábrát is (amelyek a függelékben tekinthetők meg). Egy táblázattal is demonstrálom a logikák funkcióinak összehasonlítását.

A hatodik fejezet a bemutatott öt mesterséges intelligenciával kapcsolatos szimulációkat tartalmazza. Ebben a részben különféle méréseket végzek annak érdekében, hogy megvizsgáljuk, hogy miként változik a játszmák felépítése a különböző algoritmusok egymás ellen való játéka során. Továbbá részletezem, hogy miként módosítottam a logikák működését az korábban elkészítettekhez képest, valamint, hogy ezeknek a módosításoknak milyen hatása volt a hatékonyságukra. Összegzésképpen pedig két táblázattal szemléltetem a különböző AI-ok egymással szembeni hatékonyságát.